%!TEX root = main.tex
\chapter{Análisis y conclusiones}
En este capítulo se presentan los resultados obtenidos en el desarrollo de esta
memoria.
En la sección~\ref{sec:datos} se muestra información y estadísticas de los datos
analizados como son las fechas y tipos de consultas, los \tt{endpoint} más
utilizados y otra información relevante del proyecto Bio2RDF.

La sección~\ref{sec:res} presenta los resultados obtenidos tanto de la
extracción y creación del grafo RDF como del cálculo de su centralidad. Con
estos datos se hacen los análisis y las comparaciones pertinentes.

Por último, en la sección~\ref{sec:con}, se enuncian las conclusiones generales
obtenidas junto a una comparación de ellas con lo que se espera del proyecto
Bio2RRDf. Además se evidencian los problemas detectados y se genera una lista de
posibles mejoras y trabajo futuro con respecto a este tema.


\section{Datos analizados}\label{sec:datos}
Para el análisis llevado a cabo en este trabajo se dispuso de 12Gb de consultas
almacenadas en 57.016 archivos de registros obtenidos del proyecto Bio2RDF. Las
consultas fueron efectuadas entre el 05 de mayo del 2013 hasta el 18 de
septiembre del 2015. La figura~\ref{fig:dates} muestra una gráfica de la
distribución de consultas realizadas al proyecto en este periodo. Tanto el mes
inicial como el final no presentan la totalidad de las consultas debido a que
los registros comenzaron y terminaron a mitad de mes.

\begin{figure}[ht]
  \begin{tikzpicture}
    \begin{axis}[
        xlabel=Fecha (año-mes), ylabel=Número de consultas,
        xticklabel style={rotate=90,anchor=near xticklabel},
        width=\textwidth,height=6cm,compat=1.9,
        date coordinates in=x,date ZERO=2013-05-01,
        ymin=0,ymax=1500000, xticklabel=\year-\month,
        xmin=2013-04-01,xmax=2015-10-01]
      \addplot table [x=date,y=value,col sep=comma]{data/mdates.csv};
    \end{axis}
  \end{tikzpicture}
  \caption{Fechas de las consultas.}\label{fig:dates}
\end{figure}

Los registros disponen de una total de 12.881.518 consultas hechas por 9.818 IPs
diferentes, las cuales realizaron entre 1 y 2.831.912 peticiones cada una.

En la figura~\ref{fig:ips} se muestra la cantidad de IPs que realizan hasta
cierto número de consultas.
Como podemos ver en ella, la mayoría de las IPs efectuó entre 1 y 100 consultas,
pero su aporte al total es bajo (menos de 1\%), de hecho, las 23 IPs con mayor
cantidad de consultas (más de $10^6$) aportan cerca del 80\% del total, el
detalle de estas IPs puede ser visto en la tabla~\ref{tab:ips}.

\begin{figure}[ht]
  \begin{tikzpicture}
    \begin{axis}[ybar, ymin=0, ymax=4500,
        xlabel=Número de consultas, ylabel=Número de IPs,compat=1.9,
        width=\textwidth,height=6cm,
        xtick=data,
        xticklabels={{$1$},{$10$},{$10^2$},{$10^3$},{$10^4$},{$10^5$},{$10^6$},{$10^7$}},
        nodes near coords,
        nodes near coords align={vertical}]
    \addplot table [x expr=\coordindex,y=value,col sep=comma]{data/ip.csv};
    \end{axis}
  \end{tikzpicture}
  \caption{Cantidad de consultas por IP.}\label{fig:ips}
\end{figure}

\begin{table}[ht]
  \centering
  \begin{tabular}{|r|l|l|l|} \hline
    \bf{Consultas} & \bf{IP} & \bf{Pais} & \bf{Instituación} \\\hline
    121646  & 150.214.40.112  & España         
                   & Centro Informatico Cientifico de Andalucia\\\hline
    129006  & 37.6.165.5      & Grecia         
                   & Desconocido\\\hline
    134178  & 79.107.219.216  & Grecia         
                   & Desconocido\\\hline
    141036  & 134.160.214.42  & Japón          
                   & RIKEN\\\hline
    143496  & 134.117.221.16  & Canadá         
                   & Carleton University\\\hline
    150236  & 155.185.49.66   & Italia         
                   %& Universita Degli Studi Di Modena E Reggio Emilia\\\hline
                   & Degli Studi Di Modena E Reggio Emilia\\\hline
    153794  & 134.117.108.151 & Canadá         
                   & Carleton University\\\hline
    166286  & 24.130.52.25    & EEUU 
                   & Desconocido\\\hline
    167895  & 134.117.108.111 & Canadá         
                   & Carleton University\\\hline
    217148  & 134.117.108.158 & Canadá         
                   & Carleton University\\\hline
    229289  & 173.178.48.100  & Canadá         
                   & Desconocido\\\hline
    232020  & 140.203.154.5   & Irlanda        
                   & National University of Ireland Galway\\\hline
    233677  & 159.90.11.58    & Venezuela      
                   & Universidad Simón Bolívar\\\hline
    238541  & 134.117.108.159 & Canadá         
                   & Carleton University\\\hline
    251789  & 146.155.115.75  & Chile          
                   & Pontificia Universidad Católica de Chile\\\hline
    259143  & 140.203.154.6   & Irlanda        
                   & National University of Ireland Galway\\\hline
    304598  & 133.11.132.151  & Japón          
                   & University of Tokyo\\\hline
    342553  & 140.203.154.11  & Irlanda        
                   & National University of Ireland Galway\\\hline
    478989  & 129.26.128.185  & Alemania       
                   & Fraunhofer-Gesellschaft\\\hline
    801417  & 129.26.131.1    & Alemania       
                   & Fraunhofer-Gesellschaft\\\hline
    1130035 & 134.117.221.14  & Canadá         
                   & Carleton University\\\hline
    1391974 & 171.65.32.83    & EEUU 
                   & Stanford University\\\hline
    2831912 & 132.203.117.5   & Canadá         
                   & Universite Laval\\\hline
  \end{tabular}
  \caption{IPs con más consultas.}\label{tab:ips}
\end{table}

En la tabla~\ref{tab:ips} además podemos notar como la mayoría de las
instituciones a las cuales pertenecen las IPs son universidades o centros de
investigación. 
%TODO: complementar.

%El tamaño de las consultas 
%\begin{figure}[ht]
%  \begin{tikzpicture}
%    \begin{axis}[
%        width=\textwidth,height=6cm,compat=1.9,
%        ymin=0]
%      \addplot table [y=size,col sep=comma]{data/size.csv};
%    \end{axis}
%  \end{tikzpicture}
%  \caption{Tamaño de las consultas.}\label{fig:date}
%\end{figure}

\begin{figure}[ht]
  \begin{tikzpicture}
    \begin{axis}[axis lines*=left, xbar, width=12cm, height=6cm, xlabel={},
      symbolic y coords={Desconocido, ERROR, SELECT, DESCRIBE, CONSTRUCT, ASK },
      ytick=data, xmin=0, xmax=0.6, nodes near coords,
      nodes near coords align={horizontal},xtick={0.1, 0.2, 0.3, 0.4, 0.5, 0.6},
      xticklabel={\pgfmathparse{\tick*100}\pgfmathprintnumber{\pgfmathresult}\%},
      point meta={x*100},
      nodes near coords={\pgfmathprintnumber\pgfplotspointmeta\%},
      nodes near coords align={horizontal}]
      \addplot[draw=blue,pattern=horizontal lines light blue] coordinates
      {(0.37,SELECT) (0.23,CONSTRUCT) (0.11,DESCRIBE)
       (0.04,ASK) (0.10,Desconocido) (0.14,ERROR)};

      \node[red,left] at (axis cs:0.6,SELECT)      {4.827.452};
      \node[red,left] at (axis cs:0.6,CONSTRUCT)   {3.021.587};
      \node[red,left] at (axis cs:0.6,ASK)         {560.296};
      \node[red,left] at (axis cs:0.6,ERROR)       {1.814.013};
      \node[red,left] at (axis cs:0.6,DESCRIBE)    {1.376.450};
      \node[red,left] at (axis cs:0.6,Desconocido) {1.281.720};
    \end{axis}
  \end{tikzpicture}
  \caption{Tipo de consultas realizadas.}\label{fig:qtype}
  \vspace{-.2cm}
  \caption*{En rojo el total de consultas por tipo.}
\end{figure}

\section{Resultados obtenidos}\label{sec:res}
\section{Conclusiones}\label{sec:con}
