%!TEX root = main.tex

\chapter{Introducción}

\section{Identificación del problema}
%TODO: Agregar más intro acá
El proyecto Bio2RDF (en su versión 3) incorpora la información de 35 bases de
datos RDF con información biológica, cerca de 11.000 millones de triples que
conforman la red de data enlazada más grande de esta ciencia.

Debido al volumen de datos que se manejan en este proyecto se hace interesante
determinar cual es la información más consultada por parte de los usuarios y 
así verificar que ésta tenga el soporte adecuado por parte del modelo. 
Para ello es indispensable contar con métricas que analicen el uso de los datos
consultados y la relación entre los mismos.

Si bien en trabajos anteriores como en Hu \emph{et al.}\cite{hu2015link} se han
determinado parámetros como el grado de distribución, la simetría y la
transitividad de los enlaces entre los datos, no existe un estudio
que determine cual es el subconjunto de datos que realmente son consultados por 
los usuarios y la relación entre estos y por ello no es posible verificar que
entidades son las más importantes dentro del proyecto Bio2RDF.

En trabajos anteriores\cite{hu2015link} se ha determinado que los enlaces entre
las bases de datos de Bio2RDF presentan el fenómeno de ``mundo pequeño'' lo cual
indica que a pesar del gran número de nodos que existen en esta red de datos
generalmente es posible encontrar un camino corto entre ellos. Este fenómeno se
presenta generalmente en las redes sociales donde se puede encontrar una
relación entre dos individuos cualquiera en un número acotado de pasos.

Debido a estas características consideramos atractivo generar un calculo de
centralidad para los datos consultados por los usuarios al proyecto Bio2RDF y de
esta manera identificar cuales son las instancias más importantes dentro de esta
base de datos.

\section{Objetivos}

El objetivo de este trabajo es generar estadísticas de centralidad sobre el
subconjunto de datos consultados por los usuarios al proyecto Bio2RDF y comparar
estos resultados con el modelo del proyecto en sí.

\subsection{Objetivos específicos}
Para el logro del objetivo general se plantean los siguientes objetivos
específicos:
\begin{enumerate}
  \item
    Generar un subgrafo del proyecto Bio2RDF a través del análisis de las
    consultas SPARQL hechas al servidor por parte de los usuarios.
  \item
    Analizar el grafo generado por medio de métricas de centralidad para grafos.
  \item
    Comparar los resultados del estudio con el proyecto Bio2RDF.
\end{enumerate}
