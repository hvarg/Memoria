\begin{figure}[t]
  \centering
  \begin{subfigure}[b]{\textwidth}
    \centering
    \begin{tabular}{c}
      \lstinputlisting[language=SPARQL]{
        code/example-ntriples.nt}
    \end{tabular}
    \caption{Archivo N-Triples a convertir.}
    \label{fig:nt:orig}
  \end{subfigure}
  \\[0.5cm]
  \begin{subfigure}[b]{.25\textwidth}
    \centering
    \begin{tabular}{c}
      \lstinputlisting{code/example-ntriples.nt.sg}
    \end{tabular}
    \caption{Archivo \tt{.sg}.}
    \label{fig:nt:sg}
  \end{subfigure}
  \begin{subfigure}[b]{.70\textwidth}
    \centering
    \begin{tabular}{c}
      \lstinputlisting{code/example-ntriples.nt.names}
    \end{tabular}
    \caption{Archivo de equivalencias entre nombres y ids.}
    \label{fig:nt:names}
  \end{subfigure}
  \\[0.5cm]
  \begin{subfigure}[b]{\textwidth}
    \centering
    \begin{tikzpicture}
      \begin{scope}[every node/.style={circle,thick,draw}]
        \node (0) at (0,0)    {$0$};
        \node (1) at (0,-1.5)   {$1$};
        \node (2) at (2,-1.5)    {$2$};
        \node (3) at (4,0)    {$3$};
        \node (4) at (2,0)    {$4$};
        \node (5) at (4,-1.5)    {$5$};
        \node (6) at (6,0)    {$6$};
        \node (7) at (6,-1.5)    {$7$};
      \end{scope}
      \begin{scope}[>={Stealth[black]},every edge/.style={draw=black,very thick}]
        \path[->] (0) edge (1);
        \path[->] (2) edge (1);
        \path[->] (3) edge (4);
        \path[->] (3) edge (5);
        \path[->] (3) edge (6);
        \path[->] (5) edge (7);
      \end{scope}
    \end{tikzpicture}
    \caption{Representación del grafo generado.}
    \label{fig:nt:graph}
  \end{subfigure}

  \caption{Ejemplo de creación de un grafo desde un archivo
  N-Triples.}\label{fig:nt-to-graph}
\end{figure}
