%!TEX root = main.tex

\chapter{Desarrollo de la solución}

Para el análisis de los datos revisados por los usuarios a DrugBank se contó con
los archivos de registro, en los cuales se mantienen tanto las consultas como
metadatos de ellas. Con la información disponible el proceso para calcular la
centralidad de los datos se estructura como sigue, en la sección~\ref{d:emc}
se describe como se extrajeron las consultas y como fueron modificadas para
retornar los triples utilizados, en la sección~\ref{d:cg} se presenta el modelo
utilizado para transformar la base de datos RDF a un grafo sobre el cual
calcular la centralidad y en la sección~\ref{d:cc} se explica el algoritmo
usado para este calculo.

\section{Extracción y modificación de consultas}\label{d:emc}
Los archivos de registro analizados están codificados en formato \tt{json},
cada linea es un diccionario con los siguientes elementos (entre otros):
\begin{itemize}
  \item
    Los atributos \tt{DESCRIBE}, \tt{CONSTRUCT}, \tt{SELECT} y \tt{ASK} serán
    $1$ si la consulta es de ese tipo, $0$ si no lo es y una cadena de
    caracteres vacíos si no se pudo determinar.
  \item
    El atributo \tt{ip} guarda la IP que generó la consulta.
  \item
    El atributo \tt{query} guarda la consulta completa en una cadena de
    caracteres. 
  \item
    El atributo \tt{targer\_endpoint} guarda la \tt{url} del \tt{endpoint} al
    cual se realizó la consulta.
  \item
    El atributo \tt{date} guarda la fecha y hora en las cuales se registró la
    consulta.
  \item
    El atributo \tt{response\_size} guarda la cantidad de bytes generados por la
    consulta.
  \item
    El atributo \tt{error} será \tt{true} si la consulta no fue procesada con
    éxito.
\end{itemize}

En base a las consultas guardadas en estos archivos se busca generar sus
equivalentes en \tt{CONSTRUCT} de manera que estos retornen los mismos triples
que son consultados en el servidor.

Para este proceso se ignoraron tanto las consultas con atributos \tt{ASK} o
\tt{DESCRIBE} distintos de $0$, además de aquellas con el atributo \tt{error}
como \tt{true}.

Las consultas que pasan este filtro siguen el siguiente proceso:
\begin{enumerate}
  \item
    Se analiza la cadena de caracteres y se separa en las siguientes partes:
    \begin{enumerate}
      \item
        \tt{head}: Guarda todo lo que está antes del tipo de consulta:
        \tt{PREFIX} y \tt{BASE} (también llamado prologo).
      \item
        \tt{qtype}: Guarda el tipo de consulta y sus parámetros, es decir
        \tt{SELECT} o \tt{CONSTRUCT} junto a la tabla o triples que los
        acompañan respectivamente.
      \item
        \tt{where}: Guarda la sección \tt{WHERE} de la consulta, todos los
        triples y operaciones hechas para la búsqueda están en esta sección.
      \item
        \tt{tail}: Guarda la parte final de la consulta, los modificadores de la
        solución, por ejemplo \tt{LIMIT}, \tt{ORDER} o \tt{OFFSET}.
    \end{enumerate}
  \item
    Se hace una búsqueda en la sección \tt{where} de la consulta para encontrar
    todos los triples requeridos. Este proceso recurre a la comparación de
    cadenas de caracteres, donde se separan las URIs e inmutables (comillas
    dobles o simples o tres de alguna de estas), de los separadores de triples
    (como ``\tt{.}'', ``\tt{;}'' o ``\tt{,}'') y otras operaciones (como
    \tt{FILTER},    \tt{SERVICE} o \tt{OPTIONAL}).
  \item
    La búsqueda se hace recursivamente para encontrar los triples dentro de
    estructuras más complejas como son \tt{UNION} o \tt{GRAPH}.
  \item
    Algunas consultas requieren recursos anónimos. En este caso, los
    caracteres ``\lbrack'', ``\rbrack'' o ``/'' fueron remplazados por variables
    con nombre.
  \item
    Los triples que eran parte de clausulas opcionales también fueron extraídos,
    pero guardados separadamente. Cada clausula opcional genera su propia lista
    de triples.
  \item
    Por los alcances propios de el proyecto las consultas con la operación
    \tt{SERVICE} o consultas anidadas no fueron analizadas.
\end{enumerate}

Una vez terminado este proceso se tiene tanto la consulta completa como una
lista con todos sus triples  y cero o más listas con triples
opcionales.
Con esta información se procede a generar la consulta tipo \tt{CONSTRUCT}
resultante como sigue:
\begin{enumerate}
  \item
    Para las consultas sin \tt{OPTIONAL} se modifica solo la parte \tt{qtype}
    generando un \tt{CONSTRUCT} con los triples obtenidos de la búsqueda en
    \tt{where}.
  \item
    Para las consultas con \tt{OPTIONAL} se genera la consulta descrita en el
    punto anterior y una consulta más por cada clausula opcional haciendo esta
    misma obligatoria, de esta forma, eliminando la información replicada, se 
    obtendrán la mayor cantidad de datos que la consulta puede retornar.
\end{enumerate}


El algoritmo~\ref{alg:extract} muestra el pseudo código para todo este
procedimiento. Mientras que la figura~\ref{fig:exextr} muestra un ejemplo de
su funcionamiento.

%105.158.161.3
\begin{figure}[htpb]
  \centering
  \begin{subfigure}[b]{\textwidth}
    \centering
    \begin{tabular}{c}
      \lstinputlisting[language=SPARQL]{
        code/example-consult-original.sparql}
    \end{tabular}
    \caption{Consulta original.}
    \label{fig:exextr:or}
  \end{subfigure}
  \begin{subfigure}[b]{\textwidth}
    \centering
    \begin{tabular}{c}
      \lstinputlisting[language=SPARQL]{
        code/example-consult-res1.sparql}
    \end{tabular}
    \caption{Sin opcionales.}
    \label{fig:exextr:1}
  \end{subfigure}
  \begin{subfigure}[b]{.49\textwidth}
    \centering
    \begin{tabular}{c}
      \lstinputlisting[language=SPARQL]{
        code/example-consult-res2.sparql}
    \end{tabular}
    \caption{Primer opcional.}
    \label{fig:exextr:2}
  \end{subfigure}
  \begin{subfigure}[b]{.5\textwidth}
    \centering
    \begin{tabular}{c}
      \lstinputlisting[language=SPARQL]{
        code/example-consult-res3.sparql}
    \end{tabular}
    \caption{Segundo opcional.}
    \label{fig:exextr:3}
  \end{subfigure}
  \caption{Ejemplo de transformación de consultas.}\label{fig:exextr}
\end{figure}


En la figura~\ref{fig:exextr:or} se muestra la consulta original la cual,
después del procesamiento, generará 3 consultas, todas ellas reemplazando su
tercera linea. Para la primera consulta se reemplazará por el \tt{CONSTRUCT} de
la figura~\ref{fig:exextr:1}, en la segunda por el de~\ref{fig:exextr:2} y en la
tercera por el de~\ref{fig:exextr:3}.

\begin{algorithm}
  \caption{Pseudo código para al transformación de una consulta de un grupo
  de consultas tipo \tt{CONSTRUCT} equivalente.}\label{alg:extract}
  \begin{algorithmic}[1]
    \Require Una cadena de caracteres con la consulta a analizar.
    \Ensure Una lista de cadenas de caracteres con las consultas resultantes.
    \State \it{query} \get cadena de caracteres con la consulta
    \If {\it{type}(\it{query}) = ``\tt{ASK}'' \bf{or}
         \it{type}(\it{query}) = ``\tt{DESCRIBE}''}
      \State \Return \it{None}
    \EndIf
    \State \it{head}  \get \it{get\_prologue}(\it{query})
    \State \it{qtype} \get \it{get\_query\_type}(\it{query})
    \State \it{where} \get \it{get\_where}(\it{query})
    \State \it{tail}  \get \it{get\_solution\_modifier}(\it{query})
    \State \it{triples}   \get \it{new\_list}()
    \State \it{optionals} \get \it{new\_list}()
    \ForAll{\it{sentence} \bf{in} \it{where}} 
      \If{\it{sentence} = ``\tt{OPTIONAL}''}
        \State \it{tmp} \get \it{new\_list}()
        \ForAll{\it{tr} \bf{in} \it{get\_triples}(\it{sentence})}
          \State \bf{append} \it{tr} \ra \it{tmp}
        \EndFor
        \State \bf{append} \it{tmp} \ra \it{optionals}
      \Else
        \ForAll{\it{tr} \bf{in} \it{get\_triples}(\it{sentence})}
          \State \bf{append} \it{tr} \ra \it{triples}
        \EndFor
      \EndIf
    \EndFor
    \State \it{querys} \get \it{new\_list}()
    \State \it{current\_query} \get \it{head} + ``\tt{CONSTRUCT \{}'' + 
           \it{triples} + ``\tt{\}}'' + \it{where} + \it{tail} 
    \State \bf{append} \it{current\_query} \ra \it{querys}
    \ForAll{\it{opt} \bf{in} \it{optionals}}
      \State \it{current\_query} \get \it{head} + ``\tt{CONSTRUCT \{}'' + 
             \it{triples} + \it{opt} +``\tt{\}}'' + \it{where} + \it{tail} 
      \State \bf{append} \it{current\_query} \ra \it{querys}
    \EndFor
    \State \Return \it{querys}
  \end{algorithmic}
\end{algorithm}


La implementación del algoritmo~\ref{alg:extract} fue hecha en el lenguaje de
programación \tt{python} debido a su gran simpleza a la hora de manejar cadenas
de caracteres y que el trabajo no es computacionalmente costoso para los
procesadores actuales.

El programa resultante tiene la capacidad de analizar uno o más archivos con
cualquier cantidad de lineas que corresponden a cada registro de consultas,
los resultados serán guardados en un archivo por dirección IP donde cada linea
representa una consulta generada.

Además se provee la opción de ingresar un tamaño máximo de la respuesta obtenida
(\tt{--max}) de manera que se revise el atributo \tt{response\_size} y si este
supera el máximo definido, la consulta se guarda en un archivo aparte.

Para prevenir la realización de consultas que retornen toda la base de datos se
agregó un filtro de manera de separar (en otro archivo) aquellas que solo
contienen variables en su parte \tt{CONSTRUCT}. Por ejemplo:
\begin{center}
  \tt{CONSTRUCT \{ ?a ?b ?c . \} WHERE \{ ?a ?b ?c . \}}
\end{center}

Por otro lado todas las operaciones realizadas por el programa son debidamente
regis-tradas en un archivo (\tt{log}) de manera que si ocurre algún error se
pueda determinar su causa. Este archivo además registra la fuente y el destino
de todas las consultas analizadas.

De la ejecución del programa en todos los registros se generán 
archivos que contienen todas las consultas de cada IP. Con las
consultas ya modificadas resta ejecutarlas en el \tt{endpoint} de DrugBank y
obtener los datos buscados.


\section{Creación del grafo}\label{d:cg}

\section{Calculo de centralidad}\label{d:cc}
